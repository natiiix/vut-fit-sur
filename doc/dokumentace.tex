\documentclass[a4paper,11pt]{article}

\usepackage[left=2cm, top=3cm, text={17cm, 24cm}]{geometry}
\usepackage[czech]{babel}
\usepackage[utf8]{inputenc}
\usepackage{times}
\usepackage[unicode]{hyperref}
\hypersetup{colorlinks = true, hypertexnames = false}

\begin{document}
	\begin{titlepage}
		\begin{center}
			\textsc{\Huge Vysoké učení technické v~Brně\\
				\vspace{0.4em}\huge Fakulta informačních technologií}
			
			\vspace{\stretch{0.382}}
			
			{\LARGE SUR\,--\,Strojové učení a rozpoznávání\\
				\Huge Klasifikace obličejů a řeči\\ \vspace{0.3em}}
			
			\vspace{\stretch{0.618}}
			
			{\Large \hfill Martin Kostelník (xkoste12)}\\
			{\Large \hfill Ivo Meixner (xmeixn00)}\\
			{\Large \today \hfill Adam Gajda (xgajda07)}
		\end{center}
	\end{titlepage}

	\section{Rozpoznávání obličejů}
		
		\subsection{Reprodukce výsledků}
			\begin{enumerate}
				\item Použít python 3.6.9
				\item Instalace potřebných balíků pomocí příkazu pip install -r requirements.txt
				\item Nastavit globální proměnné v souboru main.py
					\begin{itemize}
						\item TRAIN\_DIR -- Adresář se všemi trénovacími daty
						\item VALIDATION\_DIR -- Adresář se všemi validačními daty
						\item TARGET\_LABEL -- Prefix obrázků obsahujících target
						\item TEST\_DIR -- Adresář s testovacími daty
						\item LOG\_DIR -- Adresář s výsledky nástroje Keras Tuner
					\end{itemize}
				\item Připravit trénovací data. Při prvním použití je třeba spustit funkce create\_data() pro vytvoření trénovacích a validačních dat a funkci process\_test\_data() pro vytvoření testovacích dat. Tyto funkce vytvoří data v přípustném formátu pro model. Data jsou poté uloženy v souborech testing\_data.npy a validation\_data.npy. Při opakovaném použití není třeba data znovu vytvářet a stačí je načíst.
				\item Při nepřítomnosti použitého modelu je dále třeba odkomentovat příslušné řádky (LINE 187-201). Nicméně hledání nejlepšího modelu může trvat i několik hodin. Pro použití stejného modelu jako jsme použili my, je náš model dostupný v GitHub repozitáři na adrese: \url{https://github.com/natiiix/vut-fit-sur/blob/master/facial_recognition/best_model.h5}
				\item Pokud je přítomen model a jsou nachystány testovací data, stačí provést funkci recognize() a výsledky budou vypsány na standardní výstup. 
			\end{enumerate}
\end{document}